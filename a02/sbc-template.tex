\documentclass[12pt]{article}

\usepackage{sbc-template}
\usepackage{graphicx,url}
\usepackage[utf8]{inputenc}
\usepackage[brazil]{babel}
\usepackage{listings}

\lstset{
	numbers=left,
	stepnumber=1,
	firstnumber=1,
	numberstyle=\tiny,
	extendedchars=true,
	breaklines=true,
	frame=single,
	showstringspaces=false,
	xleftmargin=2.5em,
	framexleftmargin=2em,
	basicstyle=\small,
}

\renewcommand{\lstlistingname}{Código}
\renewcommand{\lstlistlistingname}{Lista de Códigos}

     
\sloppy

\title{Planejamento de um Mapeamento Sistemático da Literatura}

\author{Diogo C. T. Batista\inst{1}}

\address{Universidade Federal do Paraná (UFPR)\\
	Curitiba -- Paraná -- Brasil
	\email{diogocezar@ufpr.br}
	}


\begin{document}

\maketitle

\begin{resumo}
  Este artigo descreve a elaboração do planejamento de um mapeamento sistemático da literatura.
\end{resumo}


\section{Contexto}

O objetivo deste Mapeamento Sistmático é avaliar um subconjunto inicial de artigos relevantes ao tópico de pesquisa utilizando uma metodologia confiável, rigorosa e auditável. Com este estudo, será possível obter uma melhor compreesão dos temas que envolvem o objeto de pesquisa e principalmente a exploração de trabalhos semelhantes, e a análise de seus resultados que devem embasar e direcionar o foco de estudos do trabalho de Doutorado.

O tema da pesquisa está relacionado com a exploração de métodos, técnicas ou ferramentas de Engenharia de Software para o desenvolvimento de sistemas de monitoramento de plantações e de colaboração entre usuários no contexto da Agricultura 4.0.

A Agricultura 4.0 explora a utilização das mais recentes tecnologias computacionais tais como: a automação, a robótica agrícola, \textit{big data}, a Internet das Coisas, entre outras. Com essa exploração, busca-se uma produção agrícola eficiente e sustentável, além de ferramentas que apoiem as tomadas de decisão por parte dos envolvidos em toda a cadeia agrícola.

Entretanto, o acesso aos recursos necessários, como sensores, para a exploração dessas tecnologias não é a realidade de grande parte do setor agrícola. Adicionalmente, a resistência na adoção de novas tecnologias é um problema ainda em aberto.

Por isso, a exploração de métodos, técnicas ou ferramentas para a elaboração de um sistema que possa aproximar os agricultores da tecnologia, bem como, mostrar os benefícios da tomada de decisão de forma colaborativa são temas a serem abordados pelo trabalho.

Com este Mapeamento Sistemático busca-se analisar inicialmente, projetos que envolvam tecnologias da Agricultura 4.0 em trabalhos relacionados a interface, experiência dos usuários e colaboração entre usuários. Não serão considerados neste mapeamento o tema relacionado a monitoramento de plantações.

A seguir, estão detalhados os passos e contextos a serem utilizados no Planejamento Sistemático.

\section{Objetivo}

Como objetivos principais deste planejamento pode-se destacar:

\begin{itemize}
	\item \textbf{Analisar:} publicações Científicas;
	\item \textbf{Com o propósito de:} categorizar os trabalhos encontrados e Analisar seus conteúdos;
	\item \textbf{Com relação a:} Agricultura 4.0, Indústria 4.0, usabilidade, experiência do usuário e colaboração entre Usuários;
	\item \textbf{Do ponto de vista de:} pesquisadores da área de IHC e Engenharia de Software;
	\item \textbf{No contexto de:} pesquisas primárias sobre IHC, Agricultura 4.0, Indústria 4.0, usabilidade, experiência do usuário e colaboração entre Usuários;
\end{itemize}

\section{Questões de Pesquisa}

Nesta seção descreve-se quais são as questões utilizadas para o refinamento dos materiais encontrados. São elas:

\begin{itemize}
	\item Em qual contexto o experimento foi aplicado?
	\item Quais técnicas de usabilidade foram aplicadas?
	\item Quais técnicas de experiência do usuário foram aplicadas?
	\item Quais técnicas sobre colaboração foram aplicadas?
	\item Qual foi a metodologia utilizada para o desenvolvimento do projeto?
	\item Quais foram os métodos utilizados para os testes no experimento?
	\item Quais foram os resultados do projeto?
\end{itemize}

\section{Escopo}

Como critério para a seleção das fontes de dados utilizou-se os repositórios com maior possibilidade para a obtenção de artigos relacionados ao tema de pesquisa e Engenharia de Software.

Os repositórios explorados foram:

\begin{itemize}
	\item https://app.dimensions.ai/discover/publication
	\item https://www.tandfonline.com
	\item https://eric.ed.gov
	\item https://ieeexplore.ieee.org/Xplore/home.jsp
	\item https://dl.acm.org
\end{itemize}

\section{Idiomas}

Para os artigos explorados, buscou-se no idioma Inglês, por ser o mais utilizados em artigos da área e utilizados pela maioria dos pesquisados.

\section{Método de Busca das Publicações}

Para a definição dos termos utilizados na \textit{string} de busca, utilizou a metodologia \textit{PICO}.

\noindent\textbf{População:} agricultores de baixa renda; \\
\textbf{Intervenção:} métodos, técnicas ou ferramentas para de colaboração entre usuários no contexto da Agricultura 4.0; \\
\textbf{Comparação:} Não se aplica pois é um estudo exploratório; \\
\textbf{Resultados:} \textit{Software} usável, acessível, inclusivo e de baixo custo.

\noindent Após as definições do \textit{PICO} extraí-se as seguintes palavras-chave:

\noindent\textbf{População:} farmers, farmer, low income, small farmers, agriculture 4, agriculture 4.0, digital agriculture, precision agriculture, industry 4, industry 4.0; \\
\textbf{Intervenção:} software, method, toll, technique, framework, approach; \\
\textbf{Resultados:} usability, accessibility, inclusive, user experience, user interface, experience, interface, ux, ui, ux/ui, mobile, app, web application;

\noindent Com isso, gerou-se a string de busca detalhada no Código \ref{code:string_de_busca_gerada}.

\begin{lstlisting}[caption={String de Busca Gerada},captionpos=b,frame=single,label={code:string_de_busca_gerada}]
(
  "farmers" OR 
  "farmer" OR 
  "low income" OR 
  "small farmers" OR 
  "agriculture 4" OR 
  "agriculture 4.0" OR
  "digital agriculture" OR 
  "precision agriculture"
)
AND
(
  "software" OR
  "method" OR
  "toll" OR
  "technique" OR
  "framework" OR
  "approach"
)
AND
(
  "usability" OR
  "accessibility" OR
  "inclusive" OR
  "user experience" OR
  "user interface" OR
  "experience" OR
  "interface" OR
  "ux" OR
  "ui" OR
  "ux/ui" OR
  "mobile"
  "app" OR 
  "web application"
)
\end{lstlisting}

\section{Piloto}

Após a primeira busca, notou-se que os resultados gerados não foram satisfatórios.

Os números obtidos estão dispostos na Tabela \ref{tab:resultados_primeira_busca}.

\begin{table}[!htb]
  \centering
	\begin{tabular}{|l|l|}
		\hline
		\textbf{Ferramenta de Busca} & \textbf{Resultados} \\ \hline
		Dimensions                   & 1.808,178           \\ \hline
		Taylor                       & 2.269               \\ \hline
		Eric                         & 1.379               \\ \hline
		IEEE Xplore                  & 206                 \\ \hline
		ACM                          & 2.691               \\ \hline
		\end{tabular}
  \caption{Resultados da Primeira Busca}
  \label{tab:resultados_primeira_busca}
\end{table}

Apesar de um volume considerável de trabalhos encontrados, vários destes trouxeram o foco em palavras-chave que não tinham relação com o tema a ser explorado.

Por isso, um refinamento na busca foi aplicado, ficando com a string final demonstrada no Código \ref{code:string_de_busca_refinada}.


\begin{lstlisting}[caption={String de Busca Gerada},captionpos=b,frame=single,label={code:string_de_busca_refinada}]
(
  "agriculture 4.0" OR
  "digital agriculture" OR 
  "precision agriculture"
)
AND
(
  "software" OR
  "method" OR
  "technique" OR
  "framework" OR
  "approach"
)
AND
(
  "usability" OR
  "accessibility" OR
  "user experience" OR
  "user interface"
)
\end{lstlisting}

Após os refinamentos, os resultados obtidos estão tabulados em \ref{tab:resultados_busca_refinada}

\begin{table}[!htb]
  \centering
	\begin{tabular}{|l|l|}
		\hline
		\textbf{Ferramenta de Busca} & \textbf{Resultados} \\ \hline
		Dimensions                   & 9.578               \\ \hline
		Taylor                       & 115                 \\ \hline
		Eric                         & 1                   \\ \hline
		IEEE Xplore                  & 10                  \\ \hline
		ACM                          & 54                  \\ \hline
		\end{tabular}
  \caption{Resultados da Busca Refinada}
  \label{tab:resultados_busca_refinada}
\end{table}

Após o refinamento da \textit{string} de busca, notou-se que apesar da diminuição no número de trabalhos encontrados, em uma análise rápida feita apenas pelos títulos, os temos são mais pertinentes ao tema a ser explorado.

% \bibliographystyle{sbc}
% \bibliography{sbc-template}

\end{document}
