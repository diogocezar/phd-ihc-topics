\documentclass[12pt]{article}

\usepackage{sbc-template}
\usepackage{graphicx,url}
\usepackage[utf8]{inputenc}
\usepackage[brazil]{babel}
\usepackage{listings}
\usepackage[normalem]{ulem}
\useunder{\uline}{\ul}{}
\usepackage{longtable}
\usepackage[justification=centering]{caption}

\lstset{
	numbers=left,
	stepnumber=1,
	firstnumber=1,
	numberstyle=\tiny,
	extendedchars=true,
	breaklines=true,
	frame=single,
	showstringspaces=false,
	xleftmargin=2.5em,
	framexleftmargin=2em,
	basicstyle=\small,
}

\renewcommand{\lstlistingname}{Código}
\renewcommand{\lstlistlistingname}{Lista de Códigos}

\sloppy

\title{Execução de um Experimento sobre uso de Diretrizes de \textit{Design} e Compreensão da Informação em um Aplicativo de Ensino sobre Infecções Sexualmente Transmissíveis}

\author{Diogo C. T. Batista\inst{1}, Elissandra G. Pereira\inst{1}}

\address{Universidade Federal do Paraná (UFPR)\\
	Curitiba -- Paraná -- Brasil
	\email{diogocezar@ufpr.br,egpereira@inf.ufpr.br}
	}

\begin{document}

\maketitle

\begin{resumo}
	Este projeto descreve a execução de um experimento de validação do uso de diretrizes de \textit{design} no auxílio da compreensão da informação em um aplicativo de ensino sobre infecções sexualmente transmissíveis. O alvo do estudo é orientado à exposição de duas opções do mesmo conteúdo, na primeira expõe-se apenas um texto original com os conteúdos sobre as doenças, na segunda, apresenta o mesmo conteúdo com redesign seguindo diretrizes de \textit{design}. Para análise dos resultados, definiu-se a execução de um questionário que avalia a compreensão do conteúdo.
\end{resumo}

\section{Preparação para o Piloto}

Para a execução do piloto do experimento, foram escolhidos seis estudantes universitários para a amostra. Os participantes foram separados em dois grupos distintos entre si, no qual um grupo recebeu a versão para o experimento com a prototipação da tela original, e o outro recebeu a tela ajustada com as diretrizes de design. A distribuição foi feita de maneira aleatória, mas garantindo um número igual de participantes para ambos os testes.

Utilizou-se como ferramenta o \textit{software} Figma, para a construção dos protótipos das aplicações e o Google Forms para aplicação dos questionários.

Os participantes foram convidados por meio de aplicativo de conversa ou e-mail para responderem o experimento. Para o convite foi mencionado que se tratava de um questionário para um trabalho de mestrado e enviado o \textit{link} para o formulário do \textit{google forms}. Por mensagem também foi enviado um aviso de que as setas de navegação do programa de prototipagem não deveriam ser utilizadas, apenas o mouse na tela de celular prototipada. 

No formulário instruções mais aprofundadas foram concedidas, logo após a confirmação do TCLE. Nessa explicação, se pede para o usuário abrir o  \textit{link} do protótipo em outra aba e se familiarizar com o texto apresentado na tela do aplicativo. Também se instrui o participante a deixar o protótipo aberto, caso queira voltar a consultá-lo.

Avisos também são dados antes das primeiras perguntas sobre o conteúdo:

Responda de acordo com o que o aplicativo informa. 

Sinta-se à vontade para consultar o app para responder as perguntas caso sinta necessidade. 

Caso não encontre a informação, apenas responda que não encontrou.

\section{Etapas do Experimento}
\label{experimento}

O participante do experimento, sendo ele de qualquer grupo, recebeu um \textit{link} com o formulário online a ser respondido (google forms) contendo as informações sobre o experimento e perguntas do questionário, devidamente separados em seções:

\begin{enumerate}
\item Termo de Consentimento Livre Esclarecido (TCLE), informando, entre outros pontos, o objetivo do teste, informações sobre a segurança dos dados e o contato dos aplicadores do experimento;
\item Após a confirmação do termo os participantes foram expostos ao link do protótipo (variando entre os protótipos originais e com as diretrizes aplicadas, cada um para umm dos grupos) a explicação da tarefa a ser realizada;
\item Na sequencia, foi solicitado que respondessem um questionário a respeito da compreensão do conteúdo, composto por três perguntas de resposta aberta e três perguntas de ``sim ou não'';
\item Depois disso, pedimos que respondessem um questionário de quatro perguntas sobre a facilidade de encontrar informações e satisfação com a aparência, duas perguntas medidas com escala Likert e duas perguntas de ``sim ou não'';
\item Por último, solicitamos que respondessem três perguntas sobre idade e formação.
\end{enumerate}

\section{Como os procedimentos foram realizados}

Para suportar o momento de isolamento atual, escolhemos uma forma de teste aplicada a distância. A plataforma utilizada foi o Google Forms, por sua facilidade de criação (na hora de elaborar o formulário) e compreensão (familiaridade do público com o formato).

Para aplicar o teste convidamos os participantes e enviamos o link do formulário contendo as etapas do experimento, detalhadas na Sessão \ref{experimento}.

\section{Descrição dos participantes}

A seguir estão tabulados os dados dos participantes que participaram do experimento.

Na Tabela \ref{tab:no_guidelines} estão as informações dos participantes que foram submetidos ao protótipo do aplicativo sem a utilização das diretrizes de \textit{design}. Na Tabela \ref{tab:guidelines} estão as informações dos participantes que foram submetidos ao protótipo do aplicativo com a utilização das diretrizes de \textit{design}.

\begin{table}[!htb]
  \centering
	\begin{tabular}{|l|l|l|}
		\hline
		\textbf{Faixa Etária} & \textbf{Formação} & \textbf{Área do Conhecimento} \\ \hline
		26-33 & Graduação Completa & Biológicas \\ \hline
		26-33 & Pós Graduação Completa & Humanas \\ \hline
		26-33 & Pós Graduação Incompleta & Humanas \\ \hline
		\end{tabular}
  \caption{Participantes sem diretrizes de \textit{design}.}
  \label{tab:no_guidelines}
\end{table}

\begin{table}[!htb]
  \centering
	\begin{tabular}{|l|l|l|}
		\hline
		\textbf{Faixa Etária} & \textbf{Formação} & \textbf{Área do Conhecimento} \\ \hline
		26-33 & Pós Graduação Completa & Exatas \\ \hline
		18-25 & Ensino Superior Incompleto & Humanas \\ \hline
		18-25 & Ensino Superior Incompleto & Exatas \\ \hline
		\end{tabular}
  \caption{Participantes com diretrized de \textit{design}.}
  \label{tab:guidelines}
\end{table}

\section{Explicação da Tarefa}
\label{ap:explicacao}

Pedimos que acesse o protótipo do aplicativo ``Papo de Adolescente'' pelo seu computador \textit{desktop} ou \textit{notebook}, através do link:

\url{https://www.figma.com/proto/SJ3w95Ygfq7s7lgqnugRj4/Prot%C3%B3tipo-%22Papo-de-Adolescente%22?node-id=1%3A3&scaling=scale-down}

Acessando o protótipo, pedimos que:

\begin{itemize}
	\item Leia e se familiarize com o conteúdo da tela;
	\item Deixe o protótipo aberto em uma aba separada, caso deseje voltar a consultá-lo.
\end{itemize}

\section{Questões Aplicadas}
\label{ap:questoes}

\subsection{Sobre o Conteúdo}

Responda de acordo com o que o aplicativo informa. 

Sinta-se à vontade para consultar o app para responder as perguntas caso sinta necessidade. 

Caso não encontre a informação, apenas responda que não encontrou.

\begin{enumerate}
	\item Quais são os sintomas da segunda fase da aids? \textit{Formato: Resposta Aberta};
	\item Você consultou o app para responder a questão anterior? \textit{Formato: Opções (Sim/Não)};
	\item Quais são os sinais e sintomas da doença inflamatória pélvica? \textit{Formato: Resposta Aberta};
	\item Você consultou o app para responder a questão anterior? \textit{Formato: Opções: Sim/Não};
	\item Quais causas de corrimento não são consideradas Infecções Sexualmente Transmissíveis? \textit{Formato: Resposta Aberta};
	\item Você consultou o app para responder a questão anterior? \textit{Formato: Opções (Sim/Não)};
\end{enumerate}

\subsection{Sobre a usabilidade e satisfação}

\begin{enumerate}
	\item Como foi realizar a tarefa de encontrar informações no aplicativo? \textit{Formato: Escala Likert};
	\item Você se sentiu satisfeito com a aparência do aplicativo? \textit{Formato: Escala Likert};
	\item Você acredita que outros universitários conseguiriam entender com facilidade o conteúdo do aplicativo? \textit{Formato: Escala Likert};
	\item Você acredita que outros universitários gostariam da aparência do aplicativo? \textit{Formato: Escala Likert};
	\item Qual é a sua formação? \textit{Formato: Lista (Ensino Superior Incompleto, Ensino Superior Completo, Pós Graduação Incompleta, Pós Graduação Completa, Doutorado Incompleto, Doutorado Completo, Outro)};
	\item Dentro de qual área do conhecimento seu curso se classifica? \textit{Formato: Lista (Exatas, Humanas, Biológicas)};
	\item Qual é a sua faixa etária? \textit{Formato: Lista (18-25, 26-33, 34-41)};
\end{enumerate}

\end{document}
