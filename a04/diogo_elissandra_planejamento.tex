\documentclass[12pt]{article}

\usepackage{sbc-template}
\usepackage{graphicx,url}
\usepackage[utf8]{inputenc}
\usepackage[brazil]{babel}
\usepackage{listings}
\usepackage[normalem]{ulem}
\useunder{\uline}{\ul}{}
\usepackage{longtable}
\usepackage[justification=centering]{caption}
\usepackage{float}

\lstset{
	numbers=left,
	stepnumber=1,
	firstnumber=1,
	numberstyle=\tiny,
	extendedchars=true,
	breaklines=true,
	frame=single,
	showstringspaces=false,
	xleftmargin=2.5em,
	framexleftmargin=2em,
	basicstyle=\small,
}

\renewcommand{\lstlistingname}{Código}
\renewcommand{\lstlistlistingname}{Lista de Códigos}

     
\sloppy

\title{Planejamento de um Experimento de Usabilidade em um Aplicativo com Informativos sobre Infecções Sexualmente Transmissíveis}

\author{Diogo C. T. Batista\inst{1}, Elissandra G. Pereira\inst{1}}

\address{Universidade Federal do Paraná (UFPR)\\
	Curitiba -- Paraná -- Brasil
	\email{diogocezar@ufpr.br,egpereira@inf.ufpr.br}
	}

\begin{document}

\maketitle

\begin{resumo}
	Este projeto descreve o planejamento de um experimento de validação do uso de diretrizes de design no auxílio da compreensão da informação em um aplicativo de ensino sobre infecções sexualmente transmissíveis. O alvo do estudo é orientado à exposição de duas opções do mesmo conteúdo, na primeira expõe-se apenas um texto original com os conteúdos sobre as doenças, na segunda, apresenta o mesmo conteúdo com redesign seguindo diretrizes de design. Para análise dos resultados, definiu-se a execução de um questionário que avalia a compreensão do conteúdo.
\end{resumo}

\section{Definição do Objetivo}

De acordo com o paradigma GQM, os objetidos deste experimento podem ser definidos como:

\begin{itemize}
	\item \textbf{ANALISAR}: Uso de diretrizes de design para material educativo digital;
	\item \textbf{COM O PROPÓSITO DE}: Avaliar;
	\item \textbf{COM RESPEITO A}: Compreensão do conteúdo;
	\item \textbf{NO PONTO DE VISTA}: Dos pesquisadores do projeto;
	\item \textbf{NO CONTEXTO DE}: Alunos universitários avaliando a compreensão do conteúdo das duas versões do protótipo.
\end{itemize}

\section{Formulação de Hipóteses}

Propõe-se a utilização de hipóteses do tipo \textit{nula} e \textit{alternativa}, sendo:

\begin{itemize}
	\item \textbf{Hipótese Nula - H0}: Não há diferença entre a compreensão do conteúdo na versão sem uso das \textit{guidelines} de design em relação a versão com o uso das \textit{guidelines};
	\item \textbf{Hipótese Alternativa - H1}: Há diferença entre a compreensão do conteúdo na versão sem uso das \textit{guidelines} de design em relação a versão com o uso das \textit{guidelines}.
\end{itemize}

\section{Seleção de Variáveis}

Utiliza-se a seleção de variáveis \textit{dependentes} e \textit{independentes}:

\begin{itemize}
	\item \textbf{Variáveis Independentes}: protótipos do aplicativo (original ou com \textit{guidelines} aplicadas); 
	\item \textbf{Variáveis dependentes}: compreensão do conteúdo, satisfação do usuário ao pesquisar o conteúdo, satisfação do usuário com a aparência da tela.
\end{itemize}

\subsection{Definição da coleta e cálculo das variáveis dependentes}

A coleta das variáveis dependentes se fará através de um questionário. O questionário será composto por perguntas relacionadas a compreensão do conteúdo e satisfação do usuário. Na eficácia, será calculada a quantidade de pessoas que acertaram questões sobre o conteúdo. Na satisfação, será considerada a quantidade de pessoas que gostaram da diagramação do conteúdo e o grau de dificuldade que tiveram em encontrar as informações, ambos medidos através de uma escala Likert.

\section{Especificação do design do estudo}

A definição das diretrizes foi baseada no artigo de \cite{JIN2013248}, no qual, argumenta-se que para que para um bom \textit{design} para ensino digital, pode-se ajustar o conteúdo utilizando uma série de diretrizes.

Utilizou-se para o projeto a prototipação de uma tela chamada ``Conhecendo as ISTs'' a prototipagem tanto da réplica da tela original, quanto da tela revisada com as diretrizes foi feita pelo programa Figma.

Das diretrizes propostas por \cite{JIN2013248}, foram selecionadas três para serem aplicadas no redesign da tela, deixando de fora aquelas que exigiam um conjunto de telas, aquelas que necessitavam de ferramentas de animação que o programa utilizado para prototipagem não disponibiliza e aquelas que já foram usadas na tela original do aplicativo.

\subsection{Diretrizes selecionadas}

\subsubsection{A primeira diretriz selecionada}

\begin{quote}
	Use space to separate the paragraphs, subsections, and chapters from one another (Hartley, 1985)
\end{quote}

\begin{figure}[H]
  \centering
  \includegraphics[width=30em]{images/image_1.png}
  \caption{Exemplo de utilização da primeira diretriz}
  \label{fig:img1}
\end{figure}

É possível notar na Imagem \ref{fig:img1}, a tela sem espaço para separação dos parágrafos e sessões (original, à esquerda) e com o espaçamento aplicado (revisada, à direita).

\subsubsection{A segunda diretriz selecionada}

\begin{quote}
	Use color to help users understand what does and does not go together (Leavitt \& Shneiderman, 2006)
\end{quote}

\begin{figure}[H]
  \centering
  \includegraphics[width=30em]{images/image_2.png}
  \caption{Exemplo de utilização da segunda diretriz}
  \label{fig:img2}
\end{figure}

É possível notar na Imagem \ref{fig:img2}, a tela sem uso de cor (original, à esquerda) e com uso de cor (revisada, à direita).

Foram utilizadas as cores preto, azul e rosa. A maior parte do texto foi mantida em preto, cor original, para permanecer a maior fidelidade ao design original do aplicativo e evitar excesso de interferências na hora de testar as diretrizes. As cores adicionais, azul e rosa, seguem a identidade gráfica existente do aplicativo.

\begin{itemize}
\item \textbf{Azul}: Usado para marcar os títulos/palavras chave e avisos de grande importância.
\item \textbf{Rosa}: Usado para destacar os sintomas.
\end{itemize}

\subsubsection{A terceira diretriz selecionada}

\begin{quote}
	Design summary pages at the end of e-learning lessons (Alessi \& Trollip, 2001)
\end{quote}

\begin{figure}[H]
  \centering
  \includegraphics[width=15em]{images/image_3.png}
  \caption{Exemplo de utilização da terceira diretriz}
  \label{fig:img3}
\end{figure}

Nota-se na Imagem \ref{fig:img3}, o resumo das palavras-chave/títulos ao final da tela.

Foram selecionados os títulos de cada doença como palavras-chave para resumir o conteúdo ao final da tela. Adicionou-se também a possibilidade de clicar na palavra e rolar automaticamente para posição da tela em que o conteúdo referente a palavra-chave aparece.

O experimento será realizado em um formato \textit{between group}. Com essa estratégia, os participantes serão divididos em dois grupos. Cada grupo será exposto a uma versão diferente do aplicativo. Com issom garante-se a maior imparcialidade no experimento, visto que em uma outra possível abordagem, do tipo \textit{within group}, que definiria que cada participante seria exposto a ambas versões, o usuário poderia aprender com uma versão do aplicativo, tendo maior facilidade na avaliação da segunda versão que testasse;

\section{Seleção de participantes e ambiente/local onde o estudo deve ser realizado}

Foram escolhidos como amostra seis estudantes universitários. O público possui um grau de estudo elevado e caso encontre dificuldades com o sistema, isso impactará que usuários mais leigos possivelmente encontrarão problemas também. 

É importante ressaltar contudo que o oposto não se faz realidade: uma facilidade desse público não necessariamente reflete na facilidade de outros públicos.

Devido às limitações da quarentena, o estudo será completamente virtual, não necessitando de um ambiente específico para sua realização.

\section{Definição de Instrumentação}

A instrumentação para a pesquisa é dividida em: Termo de Consentimento Livre e Esclarecido (TCLE) e questionário de compreensão do conteúdo, satisfação, grau e área de formação acadêmica e idade do participante.

O Termo e o questionário foram enviados aos participantes por meio de um formulário do google. Foram montados dois formulários diferentes, diferindo apenas o link do protótipo. Em um formulário é apresentado um protótipo que replica a tela original, e no outro um protótipo adaptado com as diretrizes. Metade do grupo recebeu um formulário, a outra metade o outro.

O termo de Consentimento Livre e Esclarecido tem como função explicar o motivo do experimento para o participante, assim como seus direitos de pedir ajuda, parar e remover sua participação quando quiser, e avisar que os dados serão anônimos e utilizados para fins acadêmicos.

O questionário foi dividido em quatro sessões, com um total de treze perguntas.

A primeira sessão apresenta o link para o protótipo e as instruções para abri-lo e se familiarizar com o conteúdo. A segunda sessão tem três perguntas sobre o conteúdo do aplicativo, com resposta aberta e obrigatória, cada uma dessas perguntas é diretamente seguida por uma questão de ``sim ou não'' indagando se o participante consultou novamente o material para responder. 

A terceira parte, sobre usabilidade e satisfação, tem duas questões sobre a percepção individual do participante sobre o aplicativo, medidas por escalas Likert, e outras duas questões sobre o que os participantes pensam que outros universitários achariam do aplicativo, com respostas de ``sim ou não''. 

Na quarta e última sessão, foram feitas três perguntas, uma delas sobre o grau de formação universitária (graduação, mestrado ou doutorado, completos ou incompletos), uma pergunta sobre a área do conhecimento (exatas, humanas ou biológicas) e uma pergunta sobre faixa etária.

\section{Avaliação das ameaças à validade}

O público utilizado na aplicação da pesquisa pode apresentar certo viés por serem pessoas com alguma experiência na área de Interação Humano Computador, além do tema abordado não ser um assunto totalmente novo para os entrevistados.

Como ameaça também está a possibilidade dos participantes terem conhecimento prévio sobre o assunto e responderem de acordo com a memória e não o que leram no aplicativo.

Uma outra ameaça pode ocorrer caso tamanho da amostra não se mostre numeroso. Nesse caso as diferenças individuais terão grande impacto no resultado, que pode não ser expressivo.

Outro ponto importante é que pelo fato da avaliação ser realizada em formato virtual, existe a possibilidade de fraudes, ou seja, as pessoas são passíveis de responder os enunciados sem uma fidelidade de sentimento, entendimento ou emoção.

Por fim, como os testes serão realizados em um ambiente sem controle, pode-se apresentar variações pois a aplicação em questão pode ser apresentada tanto em smartphones quanto em monitores de diferentes resoluções, fazendo com que possivelmente usuários de mesmo background tenham experiências diferentes.

\bibliographystyle{sbc}
\bibliography{sbc-template}

\end{document}
